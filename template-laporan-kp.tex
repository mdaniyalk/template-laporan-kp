%-------------------------------------------------------------------------------
%                      Template Naskah Laporan Kerja Praktik
%               	     Berdasarkan format DTETI FT UGM
% 						   modified by @mdaniyalk 2024
% 				    originated from Template Naskah Skripsi
% 						    (c) @gunturdputra 2014
%-------------------------------------------------------------------------------

%Template pembuatan naskah laporan kerja praktik.
\documentclass{jtetikp}

%Untuk prefiks pada daftar gambar dan tabel
\usepackage[titles]{tocloft}
\renewcommand\cftfigpresnum{Gambar\  }
\renewcommand\cfttabpresnum{Tabel\   }

%Untuk hyperlink dan table of content
\usepackage{hyperref}
\newlength{\mylenf}
\settowidth{\mylenf}{\cftfigpresnum}
\setlength{\cftfignumwidth}{\dimexpr\mylenf+2em}
\setlength{\cfttabnumwidth}{\dimexpr\mylenf+2em}

%Untuk Bold Face pada Keterangan Gambar
\usepackage[labelfont=bf]{caption}

%Untuk caption dan subcaption
\usepackage{caption}
\usepackage{subcaption}


%-----------------------------------------------------------------
%Disini awal masukan untuk data laporan kerja praktik
%-----------------------------------------------------------------
\titleind{JUDUL}

\fullname{Muchammad Daniyal Kautsar}

\idnum{21/479067/TK/52800}

\approvaldate{dd mmmm yyyy}

\degree{Sarjana Teknologi Informasi}

\yearsubmit{2024}

\program{Teknologi Informasi}

\dept{Teknik Elektro dan Teknologi Informasi}

\firstsupervisor{ NAMA DOSEN}
\firstnip{NIP}


%-----------------------------------------------------------------
%Disini akhir masukan untuk data proposal laporan kerja praktik
%-----------------------------------------------------------------

\begin{document}

\cover

\approvalpage

%-----------------------------------------------------------------
%Disini awal masukan Bukti KP
%-----------------------------------------------------------------
\proof
\includegraphics[width=0.8\textwidth]{gambar/buktikp.png}

%-----------------------------------------------------------------
%Disini awal masukan untuk Prakata
%-----------------------------------------------------------------
\preface
Assalamu'alaikum Wr. Wb.

\vspace{0.5cm}

Puji syukur penulis panjatkan ke hadirat Allah SWT karena hanya dengan rahmat dan hidayah-Nya, Laporan Kerja Praktik ini dapat terselesaikan tanpa halangan berarti. Keberhasilan dalam menyusun laporan Kerja Praktik ini tidak lepas dari bantuan berbagai pihak yang mana dengan tulus dan ikhlas memberikan masukan guna sempurnanya laporan Kerja Praktik ini. Oleh karena itu dalam kesempatan ini, dengan kerendahan hati penulis mengucapkan terima kasih kepada:

\begin{enumerate}
\item{1.	Bapak Prof. Ir. Hanung Adi Nugroho, S.T., M.E., Ph.D., IPM., SMIEEE., selaku Ketua Departemen Teknik Elektro dan Teknologi Informasi Fakultas Teknik Universitas Gadjah Mada.}
\item{Bapak .... selaku dosen pembimbing kerja praktik yang telah memberikan banyak bantuan, bimbingan, serta arahan dalam laporan ini,}
\item{Cantumkan pihak-pihak lain yang ingin anda berikan ucapan terimakasih.}
\end{enumerate}

Penulis menyadari bahwa penyusunan Tugas Akhir ini jauh dari sempurna. Kritik dan saran dapat ditujukan langsung pada e-mail atau \emph{mention} langsung pada akun \emph{twitter} saya. Akhir kata penulis mohon maaf yang sebesar-besarnya apabila ada kekeliruan di dalam penulisan Tugas Akhir ini.

\vspace{0.5cm}

Wassalamu'alaikum Wr. Wb.

\begin{tabular}{p{7.5cm}c}
&Yogyakarta, dd month 2024\\
&\\
&\\
&\textbf{Penulis}
\end{tabular}

%-----------------------------------------------------------------
%Disini akhir masukan untuk muka laporan kerja praktik
%-----------------------------------------------------------------
\tableofcontents
\addcontentsline{toc}{chapter}{DAFTAR ISI}
\listoftables
\addcontentsline{toc}{chapter}{DAFTAR TABEL}
\listoffigures
\addcontentsline{toc}{chapter}{DAFTAR GAMBAR}

%-----------------------------------------------------------------
%Disini awal masukan untuk Bab
%-----------------------------------------------------------------
%!TEX root = ./template-skripsi.tex
%-------------------------------------------------------------------------------
% 								BAB I
% 							LATAR BELAKANG
%-------------------------------------------------------------------------------

\chapter{PENDAHULUAN}

\section{Latar Belakang Masalah}
Lorem ipsum dolor sit amet, consectetuer adipiscing elit, sed diam nonummy nibh euismod tincidunt ut laoreet dolore magna aliquam erat volutpat. Ut wisi enim ad minim veniam, quis nostrud exerci tation ullamcorper suscipit lobortis nisl ut aliquip ex ea commodo consequat. Duis autem vel eum iriure dolor in hendrerit in vulputate velit esse molestie consequat, vel illum dolore eu feugiat nulla facilisis at vero eros et accumsan et iusto odio dignissim qui blandit praesent luptatum zzril delenit augue duis dolore te feugait nulla facilisi. Nam liber tempor cum soluta nobis eleifend option congue nihil imperdiet doming id quod mazim placerat facer possim assum.\cite{wibowo2013wireless,Raluca2008}.

\section{Rumusan Masalah}
Habeo perfecto in sea. Ea deleniti gloriatur pri, paulo mediocrem incorrupte sea ei. Ad mollis scripta per. Incorrupte sadipscing ne mel. Mel ex nonumy malorum epicurei. Ne per tota mollis suscipit. Ullum labitur vim ut, ea dicit eleifend dissentias sit. Duis praesent expetenda ne sed. Sit et labitur albucius elaboraret. Ceteros efficiantur mei ad. Hendrerit vulputate democritum est at, quem veniam ne has, mea te malis ignota volumus.


\section{Batasan Masalah}
Batasan masalah pada penelitian ini adalah:
\begin{enumerate}
\item Penelitian ini difokuskan pada interoperabilitas beberapa \emph{vendor} WSN dan protokol Internet.
\item Tipe WSN yang digunakan dalam penelitian ini dibatasi dua buah.
\item Pengujian yang dilakukan hanya sebatas eksperimen dalam lingkup laboratorium.
\item Purwarupa yang dihasilkan akan diimplementasikan pada sebuah \emph{Access Point} (AP).
\end{enumerate}


\section{Tujuan Penelitian}
Eros reprimique vim no. Alii legendos volutpat in sed, sit enim nemore labores no. No odio decore causae has. Vim te falli libris neglegentur, eam in tempor delectus dignissim, nam hinc dictas an.


\section{Manfaat Penelitian}
Pro omnium incorrupte ea. Elitr eirmod ei qui, ex partem causae disputationi nec. Amet dicant no vis, eum modo omnes quaeque ad, antiopam evertitur reprehendunt pro ut. Nulla inermis est ne. Choro insolens mel ne, eos labitur nusquam eu, nec deserunt reformidans ut. His etiam copiosae principes te, sit brute atqui definiebas id.




\section{Keaslian Penelitian}
Et affert civibus has. Has ne facer accumsan argumentum, apeirian hendrerit persequeris pro ex. Suscipit vivendum sensibus mea at, vim ei hinc numquam, at dicit timeam dissentiet mel. At patrioque intellegebat sea, error argumentum dissentias sea in.


\section{Sistematika Penulisan}
No per amet modo comprehensam, duo dolor dignissim ex, ancillae corrumpit intellegam vix te. Mel utinam signiferumque no, ex nec accusam accumsan. Et per inermis posidonium, qui et ornatus epicuri pertinax. In homero commodo usu, vel te habemus fuisset, id nec periculis sententiae efficiendi. Oblique sanctus intellegat at cum.


% Baris ini digunakan untuk membantu dalam melakukan sitasi
% Karena diapit dengan comment, maka baris ini akan diabaikan
% oleh compiler LaTeX.
\begin{comment}
\bibliography{daftar-pustaka}
\end{comment}


%!TEX root = ./template-skripsi.tex
%-------------------------------------------------------------------------------
%                            BAB II
%               TINJAUAN PUSTAKA DAN DASAR TEORI
%-------------------------------------------------------------------------------

\chapter{PROFIL PERUSAHAAN}                

\section{Tinjauan Pustaka}
  Lorem ipsum is a pseudo-Latin text used in web design, typography, layout, and printing in place of English to emphasise design elements over content. It's also called placeholder (or filler) text. It's a convenient tool for mock-ups. It helps to outline the visual elements of a document or presentation, eg typography, font, or layout. Lorem ipsum is mostly a part of a Latin text by the classical author and philospher Cicero. Its words and letters have been changed by addition or removal, so to deliberately render its content nonsensical; it's not genuine, correct, or comprehensible Latin anymore. While lorem ipsum's still resembles classical Latin, it actually has no meaning whatsoever. As Cicero's text doesn't contain the letters K, W, or Z, alien to latin, these, and others are often inserted randomly to mimic the typographic appearence of European languages, as are digraphs not to be found in the original. \cite{DaSilvaCampos2011}

\section{Landasan Teori}
  \subsection{\LaTeX}
    Ne per tota mollis suscipit. Ullum labitur vim ut, ea dicit eleifend dissentias sit. Duis praesent expetenda ne sed. Sit et labitur albucius elaboraret. Ceteros efficiantur mei ad. Hendrerit vulputate democritum est at, quem veniam ne has, mea te malis ignota volumus.

    Eros reprimique vim no. Alii legendos volutpat in sed, sit enim nemore labores no. No odio decore causae has. Vim te falli libris neglegentur, eam in tempor delectus dignissim, nam hinc dictas an.

    Pro omnium incorrupte ea. Elitr eirmod ei qui, ex partem causae disputationi nec. Amet dicant no vis, eum modo omnes quaeque ad, antiopam evertitur reprehendunt pro ut. Nulla inermis est ne. Choro insolens mel ne, eos labitur nusquam eu, nec deserunt reformidans ut. His etiam copiosae principes te, sit brute atqui definiebas id.


      \begin{figure}[H]
        \centering
          \includegraphics{gambar/wsn}
          \caption{Jaringan sensor nirkabel.}
          \label{wsn}
      \end{figure}


  \subsection{Sublime Text}
    Et affert civibus has. Has ne facer accumsan argumentum, apeirian hendrerit persequeris pro ex. Suscipit vivendum sensibus mea at, vim ei hinc numquam, at dicit timeam dissentiet mel. At patrioque intellegebat sea, error argumentum dissentias sea in.

    Quo no atqui omnesque intellegat, ne nominavi argumentum quo. Eum ei purto oporteat dissentiet, soleat utamur an sit. Et assum dicam interpretaris quo. Cetero alterum ea vel, no possit alterum utroque nec. His fuisset quaestio ad. Has eu tritani incorrupte consequuntur, esse aliquip nec ne.

% Baris ini digunakan untuk membantu dalam melakukan sitasi
% Karena diapit dengan comment, maka baris ini akan diabaikan
% oleh compiler LaTeX.
\begin{comment}
\bibliography{daftar-pustaka}
\end{comment}


\include{bab3}

\include{bab4}

\include{bab5}

%-----------------------------------------------------------------
%Disini akhir masukan Bab
%-----------------------------------------------------------------


%-----------------------------------------------------------------
% Disini awal masukan untuk Daftar Pustaka
% - Daftar pustaka diambil dari file .bib yang ada pada folder ini
%   juga.
% - Untuk memudahkan dalam memanajemen dan menggenerate file .bib
%   gunakan reference manager seperti Mendeley, Zotero, EndNote,
%   dll.
%-----------------------------------------------------------------
\bibliography{IEEEabrv,daftar-pustaka}
\addcontentsline{toc}{chapter}{DAFTAR PUSTAKA}
%-----------------------------------------------------------------
%Disini akhir masukan Daftar Pustaka
%-----------------------------------------------------------------

\end{document}