%!TEX root = ./template-skripsi.tex
%-------------------------------------------------------------------------------
% 								BAB I
% 							LATAR BELAKANG
%-------------------------------------------------------------------------------

\chapter{PENDAHULUAN}

\section{Latar Belakang Masalah}
Lorem ipsum dolor sit amet, consectetuer adipiscing elit, sed diam nonummy nibh euismod tincidunt ut laoreet dolore magna aliquam erat volutpat. Ut wisi enim ad minim veniam, quis nostrud exerci tation ullamcorper suscipit lobortis nisl ut aliquip ex ea commodo consequat. Duis autem vel eum iriure dolor in hendrerit in vulputate velit esse molestie consequat, vel illum dolore eu feugiat nulla facilisis at vero eros et accumsan et iusto odio dignissim qui blandit praesent luptatum zzril delenit augue duis dolore te feugait nulla facilisi. Nam liber tempor cum soluta nobis eleifend option congue nihil imperdiet doming id quod mazim placerat facer possim assum.\cite{wibowo2013wireless,Raluca2008}.

\section{Rumusan Masalah}
Habeo perfecto in sea. Ea deleniti gloriatur pri, paulo mediocrem incorrupte sea ei. Ad mollis scripta per. Incorrupte sadipscing ne mel. Mel ex nonumy malorum epicurei. Ne per tota mollis suscipit. Ullum labitur vim ut, ea dicit eleifend dissentias sit. Duis praesent expetenda ne sed. Sit et labitur albucius elaboraret. Ceteros efficiantur mei ad. Hendrerit vulputate democritum est at, quem veniam ne has, mea te malis ignota volumus.


\section{Batasan Masalah}
Batasan masalah pada penelitian ini adalah:
\begin{enumerate}
\item Penelitian ini difokuskan pada interoperabilitas beberapa \emph{vendor} WSN dan protokol Internet.
\item Tipe WSN yang digunakan dalam penelitian ini dibatasi dua buah.
\item Pengujian yang dilakukan hanya sebatas eksperimen dalam lingkup laboratorium.
\item Purwarupa yang dihasilkan akan diimplementasikan pada sebuah \emph{Access Point} (AP).
\end{enumerate}


\section{Tujuan Penelitian}
Eros reprimique vim no. Alii legendos volutpat in sed, sit enim nemore labores no. No odio decore causae has. Vim te falli libris neglegentur, eam in tempor delectus dignissim, nam hinc dictas an.


\section{Manfaat Penelitian}
Pro omnium incorrupte ea. Elitr eirmod ei qui, ex partem causae disputationi nec. Amet dicant no vis, eum modo omnes quaeque ad, antiopam evertitur reprehendunt pro ut. Nulla inermis est ne. Choro insolens mel ne, eos labitur nusquam eu, nec deserunt reformidans ut. His etiam copiosae principes te, sit brute atqui definiebas id.




\section{Keaslian Penelitian}
Et affert civibus has. Has ne facer accumsan argumentum, apeirian hendrerit persequeris pro ex. Suscipit vivendum sensibus mea at, vim ei hinc numquam, at dicit timeam dissentiet mel. At patrioque intellegebat sea, error argumentum dissentias sea in.


\section{Sistematika Penulisan}
No per amet modo comprehensam, duo dolor dignissim ex, ancillae corrumpit intellegam vix te. Mel utinam signiferumque no, ex nec accusam accumsan. Et per inermis posidonium, qui et ornatus epicuri pertinax. In homero commodo usu, vel te habemus fuisset, id nec periculis sententiae efficiendi. Oblique sanctus intellegat at cum.


% Baris ini digunakan untuk membantu dalam melakukan sitasi
% Karena diapit dengan comment, maka baris ini akan diabaikan
% oleh compiler LaTeX.
\begin{comment}
\bibliography{daftar-pustaka}
\end{comment}
